\chapter{Конструкторский раздел}

%В данном разделе определенны структуры данных, описаны трёхмерные преобразования, входные данные, а также приведены схемы разрабатываемых алгоритмов.

\section{Проектируемая база данных}
\section{Описание используемого триггера}
\section{Система ролей в базе данных}



%\section{Описание структур данных}
%
%Для разработки общего алгоритма построения реалистического изображения в данной программе, необходимо ввести определения используемых структур данных.
%\begin{enumerate}
%    \item Сцена --- это перечень, включающий одну модель, один объект камеры и один источник освещения.
%    \item Модель многогранника включает следующие характеристики:
%    \begin{itemize}[label=---]
%        \item массив вершин фигуры;
%        \item массив полигонов фигуры;
%        \item массив векторов нормалей к вершинам;
%        \item цвет объекта;
%        \item матрица аффинных преобразований.
%    \end{itemize}
%    \item Камера содержит следующие параметры:
%    \begin{itemize}[label=---]
%        \item положение в пространстве;
%        \item расстояния до ближней и дальней плоскостей отсечения;
%        \item угол поля зрения в градусах;
%        \item углы поворота камеры по горизонтали и вертикали;
%        \item тангенс половины угла обзора, используемый в матрице проекции;
%        \item векторы, представляющие направление, куда смотрит камера, и направление вверх;
%        \item систему координат камеры, задаваемую тремя ортогональными векторами;
%        \item матрица преобразования, представляющая положение и ориентацию камеры;
%        \item точка в трехмерной сцене, на которую смотрит камера;
%        \item скорость перемещения и скорость вращения камеры;
%        \item временный вектор, используемый в вычислениях.
%    \end{itemize}
%    \item Источник освещения включает в себя следующие характеристики:
%    \begin{itemize}[label=---]
%        \item положение в пространстве;
%        \item направление света.
%    \end{itemize}
%\end{enumerate}
%
%
%\section{Аффинные преобразования}
%
%В предложенном алгоритме построения реалистического изображения первым шагом перед растеризацией полигона является трансформация модели в мировом пространстве, что достигается путем применения аффинных преобразований с использованием матриц.
%
%В рамках данного курсового проекта предусмотрены следующие операции над объектами.
%\begin{enumerate}
%    \item Поворот вокруг координатных осей описывается углом $\alpha$ и осью вращения.
%    Матрица поворота $R$ представлена следующим образом:
%    \begin{itemize}[label=---]
%        \item вокруг оси $OX$~(\ref{eq:ref28}):
%
%        \begin{equation}
%            \begin{matrix}
%                R =
%                \left( {\begin{array}{cccc}
%                            1 & 0 & 0 & 0\\
%                            0 & cos\alpha & sin\alpha & 0\\
%                            0 & -sin\alpha & cos\alpha & 0\\
%                            0 & 0 & 0 & 1\\
%                \end{array} } \right);
%                \label{eq:ref28}
%            \end{matrix}
%        \end{equation}
%
%        \item вокруг оси $OY$~(\ref{eq:ref29}):
%
%        \begin{equation}
%            \begin{matrix}
%                R =
%                \left( {\begin{array}{cccc}
%                            cos\alpha & 0 & -sin\alpha & 0\\
%                            0 & 1 & 0 & 0\\
%                            sin\alpha & 0 & cos\alpha & 0\\
%                            0 & 0 & 0 & 1\\
%                \end{array} } \right);
%                \label{eq:ref29}
%            \end{matrix}
%        \end{equation}
%
%        \item вокруг оси $OZ$~(\ref{eq:ref30}):
%
%        \begin{equation}
%            \begin{matrix}
%                R =
%                \left( {\begin{array}{cccc}
%                            cos\alpha & sin\alpha & 0 & 0\\
%                            -sin\alpha & cos\alpha & 0 & 0\\
%                            0 & 0 & 1 & 0\\
%                            0 & 0 & 0 & 1\\
%                \end{array} } \right);
%                \label{eq:ref30}
%            \end{matrix}
%        \end{equation}
%    \end{itemize}
%    \item Перенос в трехмерном пространстве задается значения вдоль координатных осей $OX$, $OY$, $OZ$ — $d_x$, $d_y$, $d_z$ соответственно.
%    Матрица переноса $M$ представлена следующим образом~(\ref{eq:ref31}):
%    \begin{equation}
%        \begin{matrix}
%            M =
%            \left( {\begin{array}{cccc}
%                        1 & 0 & 0 & 0\\
%                        0 & 1 & 0 & 0\\
%                        0 & 0 & 1 & 0\\
%                        d_x & d_y & d_z & 1\\
%            \end{array} } \right).
%            \label{eq:ref31}
%        \end{matrix}
%    \end{equation}
%\end{enumerate}
%
%\section{Приведение к пространству камеры}
%
%Для навигации по сцене применяется камера, которая определяется положением в пространстве, пирамидой видимости и собственной системой координат, состоящей из трех ортогональных векторов.
%
%Обозначим:
%\begin{itemize}[label=---]
%    \item P --- положение камеры;
%    \item D --- вектор взгляда;
%    \item U --- вектор вверх;
%    \item R --- вектор вправо.
%\end{itemize}
%
%Переход в пространство камеры выполняется в два этапа.
%
%\begin{enumerate}
%    \item Перемещение полигона в направлении, противоположном камере, на расстояние $P$, осуществляется с использованием матрицы переноса \newline $P$~(\ref{eq:ref32}):
%    \begin{equation}
%        \begin{matrix}
%            P =
%            \left( {\begin{array}{cccc}
%                        1 & 0 & 0 & 0\\
%                        0 & 1 & 0 & 0\\
%                        0 & 0 & 1 & 0\\
%                        -P_x & -P_y & -P_z & 1\\
%            \end{array} } \right).
%            \label{eq:ref32}
%        \end{matrix}
%    \end{equation}
%    \item Преобразование полигона в систему координат камеры выполняется с использованием матрицы поворота R~(\ref{eq:ref32}):
%    \begin{equation}
%        \begin{matrix}
%            R =
%            \left( {\begin{array}{cccc}
%                        R_x & U_x & D_x & 0\\
%                        R_y & U_y & D_y & 0\\
%                        R_z & U_z & D_z & 0\\
%                        0 & 0 & 0 & 1\\
%            \end{array} } \right).
%            \label{eq:ref33}
%        \end{matrix}
%    \end{equation}
%\end{enumerate}
%
%
%\section{Перспективная проекция}
%
%После перехода в пространство камеры необходимо проецировать полигон на изображение на плоскости~\cite{lookAtMatrix}.
%В данном курсовом проекте применяется перспективная проекция.
%Обозначим:
%\begin{itemize}[label=---]
%    \item $zoom_x$ --- приближение по X;
%    \item $zoom_y$ --- приближение по Y;
%    \item $R$  --- соотношение сторон экрана;
%    \item $F$ --- расстояние от камеры до задней грани пирамиды видимости;
%    \item $N$ --- расстояние от камеры до передней грани пирамиды видимости;
%    \item $\gamma$ --- вертикальный угол обзора.
%\end{itemize}
%
%Увеличение объектов по координатам X и Y можно вычислить по формулам~(\ref{eq:ref37},~\ref{eq:ref38}):
%
%\begin{equation}
%	zoom_y = \frac {1} {tg(\frac{\gamma}{2})},
%	\label{eq:ref37}
%\end{equation}
%
%\begin{equation}
%	zoom_x = \frac {zoom_y} {R}.
%	\label{eq:ref38}
%\end{equation}
%
%Для перехода в пространство отсечения используется матрица перспективной проекции $T$~(\ref{eq:ref39})~\cite{perspectiveMatrix}:
%    \begin{equation}
%        \begin{matrix}
%            T =
%            \left( {\begin{array}{cccc}
%                        zoom_x & 0 & 0 & 0\\
%                        0 & zoom_y & 0 & 0\\
%                        0 & 0 & \frac {F+N} {F-N} & 1\\
%                        0 & 0 & \frac {-2 \cdot F \cdot N} {F-N} & 0\\
%            \end{array} } \right).
%            \label{eq:ref39}
%        \end{matrix}
%    \end{equation}
%
%
%\section{Алгоритм художника}
%На рисункe~\ref{fig:painters_alg} представлена работа алгоритма художника.
%\FloatBarrier
%\begin{figure}[h]
%	\begin{center}
%		\includegraphics[scale=0.8]{inc/painters_alg}
%		\caption{Схема реализации алгоритма художника}
%		\label{fig:painters_alg}
%	\end{center}
%\end{figure}
%\FloatBarrier
%
%\section{Алгоритм однотонной закраски}
%На рисункe~\ref{fig:simple_coloring} представлена работа алгоритма однотонной закраски.
%\FloatBarrier
%\begin{figure}[h]
%	\begin{center}
%		\includegraphics[scale=0.8]{inc/simple_coloring}
%		\caption{Схема реализации алгоритма однотонной закраски}
%		\label{fig:simple_coloring}
%	\end{center}
%\end{figure}
%\FloatBarrier
%
%
%\section{Описание входных данных}
%
%В разработанной программе входные данные подаются в виде файла с расширением $.obj$.
%В этом файле, помимо координат вершин, можно передавать информацию о текстурах и нормалях.
%
%\begin{enumerate}
%    \item Список вершин с координатами $(x, y, z)$: \newline
%    v 2.279999 0.000000 0.000000
%    \item Текстурные координаты $(u, v)$: \newline
%    vt 0.600000 0.500000
%    \item Координаты нормалей $(x, y, z)$: \newline
%    vn 0.9216 0.2994 0.2469
%    \item Определение поверхности задаётся в формате $i_1/i_2/i_3$, где $i_1$ – индекс координаты вершины, $i_2$ – индекс координаты текстуры, $i_3$ – индекс координаты нормали: \newline
%    f 1/1/1 2/2/1 3/3/1
%\end{enumerate}
%
%\section{Декомпозицияя программого обеспечения}
%
%На рисункax~\ref{fig:idef0_1} --~\ref{fig:idef0_2} представленa декомпозиция программного обеспечения.
%\FloatBarrier
%\begin{figure}[h]
%	\begin{center}
%		\includegraphics[scale=0.5]{inc/idef0_1}
%		\caption{Декомпозиция программного обеспечения}
%		\label{fig:idef0_1}
%	\end{center}
%\end{figure}
%\FloatBarrier
%
%\FloatBarrier
%\begin{figure}[h]
%	\begin{center}
%		\includegraphics[scale=0.5]{inc/idef0_2}
%		\caption{Декомпозиция программного обеспечения}
%		\label{fig:idef0_2}
%	\end{center}
%\end{figure}
%\FloatBarrier
%
%\section*{Выводы}
%В этом разделе были определены структуры данных, описаны трёхмерные преобразования, входные данные, а также приведены схемы разрабатываемых алгоритмов.
