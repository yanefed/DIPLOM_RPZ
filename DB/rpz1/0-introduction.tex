\newpage
\addcontentsline{toc}{chapter}{ВВЕДЕНИЕ}
\chapter*{\makebox[1.0\linewidth]{ВВЕДЕНИЕ}}

Воздушные перевозки являются одним из наиболее предпочтительных и быстрых видов транспорта.
Самолеты --- это самые известные средства воздушного транспорта и могут использоваться не только для коммерческих/некоммерческих пассажирских перевозок, но и для транспортировки грузов и оборудования.
Kоммерческий пассажирский авиатранспорт продолжает непрерывно расти на глобальном уровне за последние десятилетия, поэтому для поддержания или увеличения качества авиаслужбы и удовлетворенности клиентов в коммерческом транспорте, необходимо удовлетворять ожидания пассажиров.
Поскольку пунктуальность оказывает влияние на предпочтение пассажиров при выборе авиалиний, крайне важно вести рейсы вовремя в коммерческом авиатранспорте~\cite{tat}.

Предварительное знание о вероятности задержки способствует информированности пассажиров и позволяет им принимать осознанные решения, включая выбор альтернативных маршрутов или распределение времени для учета возможных задержек.
Авиакомпании, в свою очередь, могут использовать такие прогнозы для управления ожиданиями пассажиров и реализации мер по снижению потенциальных негативных воздействий на их путешествия~\cite{trt}.

Целью работы является разработка базы данных для прогнозирования вероятности задержки рейса в городе--пересадки из--за технического обслуживания.

Для достижения цели необходимо выполнить следующие задачи:
\begin{enumerate}[label=\arabic*)]
    \item провести анализ предметной области;
    \item спроектировать и разработать базу данных в соответствии с поставленной задачей;
    \item выбрать средства реализации базы данных и приложения;
    \item провести сравнительный анализ работы со строками в хранимых процедурах.
\end{enumerate}