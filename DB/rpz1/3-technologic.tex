\chapter{Технологический раздел}

\section{Технологии, используемые при создании ПО}

Для выполнения данной курсовой работы был выбран язык \newline $Python$~\cite{python-lang}.

Для взаимодействия с базами данных в объектно--ориентированной парадигме было выбрано $SQLAlchemy$~\cite{sqlalchemy}.

Для работы с базой данных была выбрана СУБД $PostgreSQL$~\cite{postgresql}.

\section{Архитектура ПО}

В процессе разработки были реализованы следующие структуры:
\begin{enumerate}[label=\arabic*)]
    \item Flight --- описывает информацию о полете;
    \item Delay --- описывает информацию о задержке;
    \item Airport --- описывает информацию об аэропорте;
    \item Airline --- описывает информацию об авиакомпании;
    \item Plane --- описывает информацию о самолете;
    \item Crew --- описывает информацию о членах экипажа;
    \item Report --- описывает информацию о отчетах;
    \item System --- описывает информацию о системах;
\end{enumerate}

Также была создана функция, проверяющая наличие необходимой роли у пользователя для выполнения отправленного запроса.

\section{Реализация триггера}

В листинге~\ref{lst:trigger} представлен триггер для обновления данных в таблице system при добавлении нового отчета о самолете в таблицу report.

\begin{lstlisting}[language=SQL, label=lst:trigger, caption=Триггер для обновления данных в таблице system]
def create_trigger(db: Session):
    sql = """
    CREATE OR REPLACE FUNCTION decrease_k_coeff() RETURNS TRIGGER AS $$
    BEGIN
    UPDATE systems
    SET k_coeff = k_coeff - (random() * 10)
    WHERE plane = NEW.plane;
    RETURN NEW;
    END;
    $$ LANGUAGE plpgsql;

    CREATE TRIGGER decrease_k_coeff_trigger
    AFTER INSERT ON reports
    FOR EACH ROW EXECUTE FUNCTION decrease_k_coeff();
    """
    db.execute(text(sql))
\end{lstlisting}

\section{Реализация хранимой процедуры}
В листинге~\ref{lst:procedure} представлена хранимая процедура для получения вероятности задержки рейса.

\begin{lstlisting}[language=SQL, label=lst:procedure, caption=Хранимая процедура для получения вероятности задержки рейса]
    .........
\end{lstlisting}

\section{Реализация системы ролей}

В листинге~\ref{lst:roles} представлен код создания ролей в системе, в котором выдаются права на определенные таблицы в базе данных.

\begin{lstlisting}[language=SQL, label=lst:roles, caption=Создание ролей в системе]

    CREATE ROLE admin;
    CREATE ROLE user;
    CREATE ROLE guest;


\end{lstlisting}

\section{Примеры работы}

Для примера работы информационной системы были выполнены следующие запросы:
\begin{itemize}
    \item авторизация сотрудника;
    \item получение информации о полете;
    \item получение информации о вероятности задержки полета;
    \item добавление администратором новой записи в таблицу полетов;
    \item добавление сотрудником новой записи в таблицу отчетов;
    \item попытка добавления новой записи в таблицу аэропортов гостем;
\end{itemize}

%TODO - добавить примеры работы, хранимой процедуры
%На рисунках~\ref{fig:auth},~\ref{fig:flight},~\ref{fig:delay},~\ref{fig:add_flight},~\ref{fig:add_report},~\ref{fig:add_airport} представлены примеры работы информационной системы.