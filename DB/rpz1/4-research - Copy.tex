\chapter{Исследовательская часть}

%В данном разделе будет проведен анализ зависимости количества кадров в секунду от количества полигонов на сцене.

\section{Технические характеристики}

Технические характеристики устройства, использовавшегося для выполнения замеров, представлены ниже.

\begin{itemize}[label=---]
	\item Процессор: AMD Ryzen 7--5800H with Radeon Graphics~\cite{amd_ryzen}.
	\item Оперативная память: 16 ГБайт.
	\item Операционная система: Windows 11 Home~\cite{windows}.
	\item Частота обновления экрана: 60 Гц.
\end{itemize}

При замерах времени ноутбук был включен в сеть электропитания и был нагружен только системными приложениями.

\section{Сравнительный анализ работы со строками в хранимых процедурах, в компонентах промежуточного уровня или на клиенте}
\section*{Вывод}


%\section{Зависимость количества кадров в секунду от количества полигонов на сцене}
%
%Движение, отображаемое на экране, на самом деле представляет собой серию неподвижных изображений, представленных в быстрой последовательности.
%Наши глаза и мозг способны различать отдельные изображения (также называемые кадрами) только с частотой около 12 изображений в секунду - это означает, что мы можем отличить отдельное изображение от потока из 75 изображений в минуту.
%Еще выше — и наш мозг начинает воспринимать их как связанные, а все различия между ними как движение~\cite{fps}.
%«Кадры в секунду» (далее FPS) представляют собой количество кадров, которые отображаются в секунду на экран (независимо от частоты обновления экрана).
%
%В процессе выполнения курсового проекта была найдена связь между количеством полигонов на сцене и FPS\@.
%Для потверждения результатов исследования, необходимо провести сравнение FPS в зависимости от количества полигонов на сцене.
%Количество кадров в секунду, которое фактически видно на экране, определяется наименьшим из двух значений — количеством изображений, отправляемых на дисплей, и количеством раз, когда экран может быть обновлен.
%
%Для проведения сравнения были созданы специалные сцены, имеющие разное количество полигонов.
%
%Замеры производились на следующих параметрах:
%\begin{enumerate}
%	\item Количество полигонов на сцене;
%	\item FPS -- рассчитывается как среднее значение из 20 измерений.
%\end{enumerate}
%
%В таблице~\ref{tbl:time} приведены результаты замеров:
%\FloatBarrier
%\begin{table}[h!]
%	\caption{\centering Результаты замеров}
%	\centering
%	\begin{tabular}{ | c | c |}
%		\hline
%		Количество полигонов & FPS \\
%		\hline
%		108 & 60 \\ %torus
%		216 & 50 \\ %combined_torus
%		296 & 40 \\ %sphere
%		612 & 27 \\ %monkeyblender_2px
%		654 & 27 \\ %monkeyblender_1px
%		728 & 22 \\ %deer
%		1433 & 10 \\ %utahteapot
%		1950 & 9 \\ %utahteapot_0-5px
%		8968 & 2 \\ %Knot
%		\hline
%	\end{tabular}
%	\label{tbl:time}
%\end{table}
%\FloatBarrier
%
%На рисунке~\ref{fig:g1} представлен график зависимости количества кадров в секунду (FPS) от количества полигонов на сцене:
%
%%\begin{filecontents*}{times.csv}
%%polygons,fps
%%8968,2
%%1950,9
%%1433,10
%%728,22
%%654,27
%%612,27
%%296,40
%%216,50
%%108,60
%%\end{filecontents*}
%
%\begin{figure}[h!]
%	\centering
%	\begin{tikzpicture}[scale=1.2]
%		\begin{axis}[
%			xlabel=К-во полигонов (ед.),
%			ylabel=FPS,
%		 	grid=both,
%			xtick={0, 200, 400, 600, 800, 1000, 1250, 1500, 1750, 2000},
%			xticklabel style={/pgf/number format/fixed},
%			ytick={0, 5, 10, 15, 20, 25, 30, 35, 40, 45, 50, 55, 60},
%			yticklabel style={/pgf/number format/fixed},
%			width=14cm,
%			legend style={anchor=north}, ]
%			\addplot [
%			dashed,
%			thick,
%			draw = blue,
%			mark = --,
%			mark options = {
%				scale = 2,
%				fill = blue,
%				draw = black
%			}
%			] coordinates {
%                (108, 60)
%                (216, 50)
%                (296, 40)
%                (612, 27)
%                (654, 27)
%                (728, 22)
%                (1433, 10)
%                (1950, 9)
%            };
%			\addplot+[
%                only marks,
%                mark=*,
%                mark options={
%                    scale=1.5,
%                    fill=blue,
%                    draw=black
%                }
%            ] coordinates {
%                (108, 60)
%                (216, 50)
%                (296, 40)
%                (612, 27)
%                (654, 27)
%                (728, 22)
%                (1433, 10)
%                (1950, 9)
%            };
%		\end{axis}
%	\end{tikzpicture}
%	\caption{График зависимости FPS от количества полигонов на сцене}
%	\label{fig:g1}
%\end{figure}
%
%\newpage
%
%\section*{Выводы}
%
%Анимация с частотой 24 кадра в секунду имеет сильное размытие при движении, с частотой 30 -- кажется неестественной из-за заметной смены кадров, а с частотой 60 -- будет выглядеть более четкой и плавной из-за низкого размытия при движении.
%Количество кадров в секунду влияет на возможность работы с интерфейсом программы, так как результат взаимодействия пользователя с объектом будет виден только спустя несколько кадров.
%Было выявлено, что размер загружаемого объекта также влияет на количество полигонов - чем меньше объект, тем больше отрисованных полигонов.
%Для оптимальной работы с программым продуктом, рекоммендуется работать с оптимизированными объектами по количеству полигонов.