\addcontentsline{toc}{chapter}{ЗАКЛЮЧЕНИЕ}
\chapter*{\makebox[1.0\linewidth]{ЗАКЛЮЧЕНИЕ}}

Курсовой проект включал в себя анализ различных методов представления объектов, алгоритмов удаления скрытых линий и поверхностей, а также моделей освещения.
В работе были отмечены как преимущества, так и недостатки данных методов.

На базе результатов анализа было спроектировано программное обеспечение, предназначенное для формирования индивидуальных сцен, использующих загружаемые геометрические трёхмерные объекты.

Исследование в области зависимости количества кадров в секунду от количества полигонов на сцене выявило, что для оптимальной работы с разработанным программным продуктом, необходимо работать с оптимизированными объектами по количеству полигонов.

Цель курсовой работы была достигнута.
В ходе выполнения работы были выполнены следующие задачи:
\begin{enumerate}[label=\arabic*)]
	\item проведен анализ существующих алгоритмов компьютерной графики для создания реалистичных изображений;
	\item выбраны наиболее подходящие алгоритмы для достижения поставленной цели;
	\item спроектированы архитектуры и графического интерфейса программы;
	\item выбраны средства для реализации программного обеспечения;
	\item разработано программное обеспечение;
	\item проведено исследование разработанного программного обеспечения.
\end{enumerate}
