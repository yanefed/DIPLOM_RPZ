\chapter{Технологический раздел}
%В этом разделе осуществляется выбор инструментов для реализации, предоставляется описание классов программы и интерфейс программного обеспечения.


\section{Технологии, используемые при создании ПО}
\section{Архитектура ПО}
\section{Реализация триггера}
\section{Реализация системы ролей}
\section{Примеры работы}

%\section{Требования к программному обеспечению}
%
%Программа должна иметь графический интерфейс, предоставляющий следующий функционал:
%\begin{itemize}[label=---]
%    \item изменение положения модели многогранника в пространстве;
%    \item регулировка положения камеры и направления её обзора в пространстве.
%\end{itemize}
%
%Разработанное программное обеспечение должно соответствовать следующим требованиям:
%\begin{itemize}[label=---]
%    \item источник света создаётся при запуске программы;
%    \item допускается наличие только одной камеры в пространстве.
%\end{itemize}
%
%\section{Средства реализации программного обеспечения}
%Для выполнения данного курсового проекта был выбран язык программирования Python. Этот выбор обоснован следующими причинами:
%\begin{itemize}[label=---]
%	\item имеется опыт разработки на этом языке;
%	\item Python поддерживает объектно-ориентированный подход к разработке, что позволяет структурировать программу и создавать качественное программное обеспечение;
%	\item Python обладает широким выбором различных библиотек, которые могут помочь в реализации данного программного обеспечения.
%\end{itemize}
%
%В качестве среды разработки была выбрана PyCharm по следующим причинам:
%\begin{itemize}[label=---]
%	\item в данной среде присутстввуют встроенный отладчик, инструменты для работы с Git, а также средства для навигации по коду, автодополнения кода и контекстной подсказки.
%\end{itemize}
%
%\section{Диаграмма классов}
%
%На рисунке~\ref{fig:class_diagram} изображена диаграмма классов, использующихся в программе.
%\begin{itemize}[label=---]
%	\item SoftwareRender -- класс оконного интерфейса;
%	\item Scene -- класс, использующийся для визуализации сцены;
%	\item Light -- класс точечного источника света;
%	\item Point -- класс геометрической точки;
%	\item Mesh -- класс отображаемого объекта;
%	\item Camera -- класс, хранящий информацию о камере;
%	\item Matrix -- класс матрицы;
%	\item Vector3 -- класс трехмерного вектора;
%	\item Tools -- файл с вспомогательными функциями;
%	\item Event -- файл с функциями для обработки внешних сигналов.
%\end{itemize}
%
%\FloatBarrier
%\begin{figure}[h]
%	\begin{center}
%		\includegraphics[scale=0.9]{inc/class_diagram}
%		\caption{Диаграмма классов}
%		\label{fig:class_diagram}
%	\end{center}
%\end{figure}
%\FloatBarrier
%
%\section{Интерфейс программы}
%
%На рисунке~\ref{fig:interface} демонстрируется интерфейс программы.
%\FloatBarrier
%\begin{figure}[h]
%	\begin{center}
%		\includegraphics[scale=0.5]{inc/interface}
%		\caption{Интерфейс программы}
%		\label{fig:interface}
%	\end{center}
%\end{figure}
%\FloatBarrier
%
%Для взаимодействия с освещением объекта на сцене используется позиция курсора мыши на экране.
%Для управления камерой используются следующие клавиши:
%\begin{itemize}[label=---]
%	\item $\uparrow$ -- наклон вверх;
%	\item $\downarrow$ -- наклон вниз;
%	\item $\rightarrow$ -- поворот вправо;
%	\item $\leftarrow$ -- поворот влево;
%	\item $W$ -- перемещение вперед;
%	\item $A$ -- перемещение влево;
%	\item $S$ -- перемещение назад;
%	\item $D$ -- перемещение вправо.
%\end{itemize}
%
%
%\section{Реализация алгоритмов}
%
%В листингах 3.1 -- 3.2 представлены реализации рассматриваемых алгоритмов.
%
%\begin{lstlisting}[caption=Реализация алгоритма художника]
%	def update(self, dt, camera, light, screen, showAxis=False, fill=True,
%               wireframe=False, vertices=False, depth=True,
%               clippingDebug=False, showNormals=False, radius=2,
%               verticeColor=False, wireframeColor=(255, 255, 255),
%               ChangingColor=0, lineWidth=1):
%    camera.HandleInput(dt)
%    camera.direction = vector.Vector3(0, 0, 1)
%    camera.up = vector.Vector3(0, 1, 0)
%    camera.target = vector.Vector3(0, 0, 1)
%    camera.rotation = matrix.Matrix.rotation_x(camera.yaw)
%    camera.direction = matrix.multiplyMatrixVector(camera.target, camera.rotation)
%    camera.target = camera.position + camera.direction
%    lookAtMatrix = PointAt(camera.position,
%    camera.target, camera.up)
%    camera.viewMatrix = QuickInverse(lookAtMatrix)
%    camera.target = Vector3(0, 0, 1)
%
%    triangles = []
%    origins = []
%    for ob in self.world:
%        triangles += ob.update(screen, fill, wireframe, dt, camera, light, depth, clippingDebug, ChangingColor)
%
%    triangles.sort(key=lambda val: (val.vertex1.z + val.vertex2.z + val.vertex3.z) / 3.0)
%
%    normals_length = 250
%    normals = []
%
%    for projected in reversed(triangles):
%        origin = (projected.vertex1 + projected.vertex2 + projected.vertex3) / 3
%        line1 = projected.vertex2 - projected.vertex1
%        line2 = projected.vertex3 - projected.vertex1
%        normal = crossProduct(line1, line2) * normals_length
%        DrawTriangle(screen, projected, fill, wireframe, vertices, radius, verticeColor, wireframeColor, lineWidth)
%        origins.append(origin)
%        normals.append(normal)
%
%    if showAxis:
%        DrawAxis(screen, camera)
%
%    if showNormals:
%        # get the normal vectors
%        for i, n in enumerate(normals):
%            endPoint = origins[i] + (n)
%            pygame.draw.line(
%                screen, (0, 255, 0), origins[i].GetTuple(), endPoint.GetTuple(), 2
%            )
%\end{lstlisting}
%
%\begin{lstlisting}[caption=Реализация алгоритма однотонной закраски]
%    def compute_color_component(color):
%        color_val = color * val
%        color_val = min(255, color_val)
%        color_val = max(0, color_val)
%        return int(color_val)
%    r = compute_color_component(self.color[0])
%    g = compute_color_component(self.color[1])
%    b = compute_color_component(self.color[2])
%\end{lstlisting}
%
%%\addcontentsline{toc}{section}{Выводы}
%\section*{Выводы}
%
%В этом разделе был произведен выбор языка программирования и определен набор инструментов для реализации данного программного обеспечения.
%Также, были описаны диаграмма классов и способы взаимодействия с программой.