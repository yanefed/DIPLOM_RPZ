\usepackage{mathtext}
\usepackage[T2A]{fontenc}
\usepackage[utf8]{inputenc}
\usepackage[russian]{babel}
\usepackage{amsfonts}
\usepackage{latexsym}
\usepackage{mathtools}
\usepackage{graphicx}
\usepackage{csvsimple}
\graphicspath{}
\DeclareGraphicsExtensions{.pdf,.png,.jpg, .eps}
\usepackage[top=20mm, right = 1mm, left = 30mm, bottom = 20mm]{geometry}
\linespread{1.3}
\usepackage{color} 
\usepackage{listings}
\usepackage{pythonhighlight}
\usepackage{caption}
\captionsetup[lstlisting]{justification=raggedright,singlelinecheck=off}
\captionsetup{labelsep=endash}


\usepackage{amsmath}
\linespread{1.3}
\usepackage{geometry}
\geometry{left=30mm}
\geometry{right=15mm}
\geometry{top=20mm}
\geometry{bottom=20mm}

\usepackage{setspace}
\onehalfspacing % Полуторный интервал

\frenchspacing
\usepackage{indentfirst} % Красная строка

\usepackage{titlesec}
\titleformat{\section}
{\normalsize\bfseries}
{\thesection}
{1em}{}
\titlespacing*{\chapter}{0pt}{-30pt}{8pt}
\titlespacing*{\section}{\parindent}{*4}{*4}
\titlespacing*{\subsection}{\parindent}{*4}{*4}

\usepackage{titlesec}
\titleformat{\chapter}{\LARGE\bfseries}{\thechapter}{20pt}{\LARGE\bfseries}
\titleformat{\section}{\Large\bfseries}{\thesection}{20pt}{\Large\bfseries}

\usepackage[figurename=Рисунок]{caption}
\usepackage{placeins}


\lstset{
	literate=
	{а}{{\selectfont\char224}}1
	{б}{{\selectfont\char225}}1
	{в}{{\selectfont\char226}}1
	{г}{{\selectfont\char227}}1
	{д}{{\selectfont\char228}}1
	{е}{{\selectfont\char229}}1
	{ё}{{\"e}}1
	{ж}{{\selectfont\char230}}1
	{з}{{\selectfont\char231}}1
	{и}{{\selectfont\char232}}1
	{й}{{\selectfont\char233}}1
	{к}{{\selectfont\char234}}1
	{л}{{\selectfont\char235}}1
	{м}{{\selectfont\char236}}1
	{н}{{\selectfont\char237}}1
	{о}{{\selectfont\char238}}1
	{п}{{\selectfont\char239}}1
	{р}{{\selectfont\char240}}1
	{с}{{\selectfont\char241}}1
	{т}{{\selectfont\char242}}1
	{у}{{\selectfont\char243}}1
	{ф}{{\selectfont\char244}}1
	{х}{{\selectfont\char245}}1
	{ц}{{\selectfont\char246}}1
	{ч}{{\selectfont\char247}}1
	{ш}{{\selectfont\char248}}1
	{щ}{{\selectfont\char249}}1
	{ъ}{{\selectfont\char250}}1
	{ы}{{\selectfont\char251}}1
	{ь}{{\selectfont\char252}}1
	{э}{{\selectfont\char253}}1
	{ю}{{\selectfont\char254}}1
	{я}{{\selectfont\char255}}1
	{А}{{\selectfont\char192}}1
	{Б}{{\selectfont\char193}}1
	{В}{{\selectfont\char194}}1
	{Г}{{\selectfont\char195}}1
	{Д}{{\selectfont\char196}}1
	{Е}{{\selectfont\char197}}1
	{Ё}{{\"E}}1
	{Ж}{{\selectfont\char198}}1
	{З}{{\selectfont\char199}}1
	{И}{{\selectfont\char200}}1
	{Й}{{\selectfont\char201}}1
	{К}{{\selectfont\char202}}1
	{Л}{{\selectfont\char203}}1
	{М}{{\selectfont\char204}}1
	{Н}{{\selectfont\char205}}1
	{О}{{\selectfont\char206}}1
	{П}{{\selectfont\char207}}1
	{Р}{{\selectfont\char208}}1
	{С}{{\selectfont\char209}}1
	{Т}{{\selectfont\char210}}1
	{У}{{\selectfont\char211}}1
	{Ф}{{\selectfont\char212}}1
	{Х}{{\selectfont\char213}}1
	{Ц}{{\selectfont\char214}}1
	{Ч}{{\selectfont\char215}}1
	{Ш}{{\selectfont\char216}}1
	{Щ}{{\selectfont\char217}}1
	{Ъ}{{\selectfont\char218}}1
	{Ы}{{\selectfont\char219}}1
	{Ь}{{\selectfont\char220}}1
	{Э}{{\selectfont\char221}}1
	{Ю}{{\selectfont\char222}}1
	{Я}{{\selectfont\char223}}1
}

\usepackage[unicode,pdftex]{hyperref}
\hypersetup{hidelinks}
\usepackage{longtable}
\usepackage{graphicx}
\renewcommand{\rmdefault}{cmr}

\usepackage[justification=centering]{caption}
\usepackage{pdfpages}


\usepackage{enumitem}
\usepackage{placeins}
\usepackage{tabularx}
\usepackage{amsmath}
\usepackage{geometry}
\geometry{left=25mm}
\geometry{right=15mm}
\geometry{top=20mm}
\geometry{bottom=20mm}
\usepackage{titlesec}
\usepackage{caption}

\captionsetup[table]{justification=raggedright,singlelinecheck=off}
\captionsetup[lstlisting]{justification=raggedright,singlelinecheck=off}
\captionsetup{labelsep=endash}
\captionsetup[figure]{name={Рисунок}}

\usepackage{textgreek}
\usepackage{algpseudocode}
\usepackage{pgfplots}
\pgfplotsset{compat=1.9}
\usepackage{algorithm}
\renewcommand{\listalgorithmname}{Список алгоритмов}
\floatname{algorithm}{Алгоритм}
\usepackage{amssymb}
\titleformat{\section}
{\normalsize\bfseries}
{\thesection}
{1em}{}
\titlespacing*{\chapter}{0pt}{-30pt}{8pt}
\titlespacing*{\section}{\parindent}{*4}{*4}
\titlespacing*{\subsection}{\parindent}{*4}{*4}
\usepackage{setspace}
\usepackage{mathtools}
\usepackage{float}
\DeclarePairedDelimiter\bra{\langle}{\rvert}
\DeclarePairedDelimiter\ket{\lvert}{\rangle}
\DeclarePairedDelimiterX\braket[2]{\langle}{\rangle}{#1 \vert #2}
\onehalfspacing % Полуторный интервал
\frenchspacing
\usepackage{indentfirst} % Красная строка
\setlength\parindent{1.25cm}
\usepackage{titlesec}
\titleformat{\chapter}{\LARGE\bfseries}{\thechapter}{20pt}{\LARGE\bfseries}
\titleformat{\section}{\Large\bfseries}{\thesection}{20pt}{\Large\bfseries}
\usepackage{listings}
\usepackage{xcolor}
\lstdefinestyle{rustlang}{
	language=Rust,
	backgroundcolor=\color{white},
	basicstyle=\footnotesize\ttfamily,
	keywordstyle=\color{purple},
	stringstyle=\color{green},
	commentstyle=\color{gray},
	numbers=left,
	stepnumber=1,
	numbersep=5pt,
	frame=single,
	tabsize=4,
	captionpos=t,
	breaklines=true,
	breakatwhitespace=true,
	escapeinside={\#*}{*)},
	morecomment=[l][\color{magenta}]{\#},
	columns=fullflexible
}
\usepackage{pgfplots}
\usetikzlibrary{datavisualization}
\usetikzlibrary{datavisualization.formats.functions}
\usepackage{graphicx}
\usepackage[justification=centering]{caption} % Настройка подписей float объектов
\usepackage[unicode,pdftex]{hyperref} % Ссылки в pdf
\hypersetup{hidelinks}
\newcommand{\code}[1]{\texttt{#1}}
\usepackage{icomma} % Интеллектуальные запятые для десятичных чисел
\usepackage{csvsimple}

\usepackage{color} %use color
\definecolor{mygreen}{rgb}{0,0.6,0}
\definecolor{mygray}{rgb}{0.5,0.5,0.5}
\definecolor{mymauve}{rgb}{0.58,0,0.82}

%Customize a bit the look
\lstset{ %
	backgroundcolor=\color{white}, % choose the background color; you must add \usepackage{color} or \usepackage{xcolor}
	basicstyle=\footnotesize, % the size of the fonts that are used for the code
	breakatwhitespace=false, % sets if automatic breaks should only happen at whitespace
	breaklines=true, % sets automatic line breaking
	captionpos=b, % sets the caption-position to bottom
	commentstyle=\color{mygreen}, % comment style
	deletekeywords={...}, % if you want to delete keywords from the given language
	escapeinside={\%*}{*)}, % if you want to add LaTeX within your code
	extendedchars=true, % lets you use non-ASCII characters; for 8-bits encodings only, does not work with UTF-8
	frame=single, % adds a frame around the code
	keepspaces=true, % keeps spaces in text, useful for keeping indentation of code (possibly needs columns=flexible)
	keywordstyle=\color{blue}, % keyword style
	% language=Octave, % the language of the code
	morekeywords={*,...}, % if you want to add more keywords to the set
	numbers=left, % where to put the line-numbers; possible values are (none, left, right)
	numbersep=5pt, % how far the line-numbers are from the code
	numberstyle=\tiny\color{mygray}, % the style that is used for the line-numbers
	rulecolor=\color{black}, % if not set, the frame-color may be changed on line-breaks within not-black text (e.g. comments (green here))
	showspaces=false, % show spaces everywhere adding particular underscores; it overrides 'showstringspaces'
	showstringspaces=false, % underline spaces within strings only
	showtabs=false, % show tabs within strings adding particular underscores
	stepnumber=1, % the step between two line-numbers. If it's 1, each line will be numbered
	stringstyle=\color{mymauve}, % string literal style
	tabsize=2, % sets default tabsize to 2 spaces
	title=\lstname % show the filename of files included with \lstinputlisting; also try caption instead of title
}
%END of listing package%

\definecolor{darkgray}{rgb}{.4,.4,.4}
\definecolor{purple}{rgb}{0.65, 0.12, 0.82}

%define Javascript language
\lstdefinelanguage{JavaScript}{
	keywords={func, new, true, false, catch, return, nil, catch, switch, var, if, in, while, do, else, case, break, let},
	keywordstyle=\color{blue}\bfseries,
	ndkeywords={class, export, boolean, throw, implements, import, this},
	ndkeywordstyle=\color{darkgray}\bfseries,
	identifierstyle=\color{black},
	sensitive=false,
	comment=[l]{//},
	morecomment=[s]{/*}{*/},
	commentstyle=\color{purple}\ttfamily,
	stringstyle=\color{red}\ttfamily,
	morestring=[b]',
	morestring=[b]"
}

\lstset{
	language=JavaScript,
	extendedchars=true,
	basicstyle=\footnotesize\ttfamily,
	showstringspaces=false,
	showspaces=false,
	numbers=left,
	numberstyle=\footnotesize,
	numbersep=9pt,
	tabsize=2,
	breaklines=true,
	showtabs=false,
	captionpos=b
}


\usepackage{multirow}
\usepackage{threeparttable}
\usepackage{longtable}


\usepackage{caption}
\captionsetup[table]{justification=raggedright,singlelinecheck=off}
\captionsetup[lstlisting]{justification=raggedright,singlelinecheck=off}
\captionsetup{labelsep=endash}
\captionsetup[figure]{name={Рисунок}}

\usepackage{amsmath}
\usepackage{csvsimple}
\usepackage{enumitem}
\setenumerate[0]{label=\arabic*)} % Изменение вида нумерации списков
\renewcommand{\labelitemi}{---}

\usepackage{geometry}
\geometry{left=30mm}
\geometry{right=15mm}
\geometry{top=20mm}
\geometry{bottom=20mm}

\usepackage{titlesec}
\titleformat{\section}{\normalsize\bfseries}{\thesection}{1em}{}
\titlespacing*{\chapter}{0pt}{-30pt}{8pt}
\titlespacing*{\section}{\parindent}{*4}{*4}
\titlespacing*{\subsection}{\parindent}{*4}{*4}

\usepackage{titlesec}
\titleformat{\chapter}{\LARGE\bfseries}{\thechapter}{16pt}{\LARGE\bfseries}
\titleformat{\section}{\Large\bfseries}{\thesection}{16pt}{\Large\bfseries}

\usepackage{setspace}
\onehalfspacing % Полуторный интервал

% Список литературы
\makeatletter
\def\@biblabel#1{#1.} % Изменение нумерации списка использованных источников
\makeatother

\frenchspacing
\usepackage{indentfirst} % Красная строка после заголовка
\setlength\parindent{1.25cm}

\usepackage{ulem} % Нормальное нижнее подчеркивание
\usepackage{hhline} % Двойная горизонтальная линия в таблицах
\usepackage[figure,table]{totalcount} % Подсчет изображений, таблиц
\usepackage{rotating} % Поворот изображения вместе с названием
\usepackage{lastpage} % Для подсчета числа страниц

% Дополнительное окружения для подписей
\usepackage{array}
\newenvironment{signstabular}[1][1]{
\renewcommand*{\arraystretch}{#1}
\tabular
}{
\endtabular
}

\usepackage{listings}
\usepackage{xcolor}

% Для листинга кода:
\lstset{%
language=python,   					% выбор языка для подсветки
basicstyle=\small\sffamily,			% размер и начертание шрифта для подсветки кода
numbersep=5pt,
numbers=left,						% где поставить нумерацию строк (слева\справа)
%numberstyle=,					    % размер шрифта для номеров строк
stepnumber=1,						% размер шага между двумя номерами строк
xleftmargin=17pt,
showstringspaces=false,
numbersep=5pt,						% как далеко отстоят номера строк от подсвечиваемого кода
frame=single,						% рисовать рамку вокруг кода
tabsize=4,							% размер табуляции по умолчанию равен 4 пробелам
captionpos=t,						% позиция заголовка вверху [t] или внизу [b]
breaklines=true,
breakatwhitespace=true,				% переносить строки только если есть пробел
escapeinside={\#*}{*)},				% если нужно добавить комментарии в коде
backgroundcolor=\color{white}
}

\lstset{
literate=
{а}{{\selectfont\char224}}1
{б}{{\selectfont\char225}}1
{в}{{\selectfont\char226}}1
{г}{{\selectfont\char227}}1
{д}{{\selectfont\char228}}1
{е}{{\selectfont\char229}}1
{ё}{{\"e}}1
{ж}{{\selectfont\char230}}1
{з}{{\selectfont\char231}}1
{и}{{\selectfont\char232}}1
{й}{{\selectfont\char233}}1
{к}{{\selectfont\char234}}1
{л}{{\selectfont\char235}}1
{м}{{\selectfont\char236}}1
{н}{{\selectfont\char237}}1
{о}{{\selectfont\char238}}1
{п}{{\selectfont\char239}}1
{р}{{\selectfont\char240}}1
{с}{{\selectfont\char241}}1
{т}{{\selectfont\char242}}1
{у}{{\selectfont\char243}}1
{ф}{{\selectfont\char244}}1
{х}{{\selectfont\char245}}1
{ц}{{\selectfont\char246}}1
{ч}{{\selectfont\char247}}1
{ш}{{\selectfont\char248}}1
{щ}{{\selectfont\char249}}1
{ъ}{{\selectfont\char250}}1
{ы}{{\selectfont\char251}}1
{ь}{{\selectfont\char252}}1
{э}{{\selectfont\char253}}1
{ю}{{\selectfont\char254}}1
{я}{{\selectfont\char255}}1
{А}{{\selectfont\char192}}1
{Б}{{\selectfont\char193}}1
{В}{{\selectfont\char194}}1
{Г}{{\selectfont\char195}}1
{Д}{{\selectfont\char196}}1
{Е}{{\selectfont\char197}}1
{Ё}{{\"E}}1
{Ж}{{\selectfont\char198}}1
{З}{{\selectfont\char199}}1
{И}{{\selectfont\char200}}1
{Й}{{\selectfont\char201}}1
{К}{{\selectfont\char202}}1
{Л}{{\selectfont\char203}}1
{М}{{\selectfont\char204}}1
{Н}{{\selectfont\char205}}1
{О}{{\selectfont\char206}}1
{П}{{\selectfont\char207}}1
{Р}{{\selectfont\char208}}1
{С}{{\selectfont\char209}}1
{Т}{{\selectfont\char210}}1
{У}{{\selectfont\char211}}1
{Ф}{{\selectfont\char212}}1
{Х}{{\selectfont\char213}}1
{Ц}{{\selectfont\char214}}1
{Ч}{{\selectfont\char215}}1
{Ш}{{\selectfont\char216}}1
{Щ}{{\selectfont\char217}}1
{Ъ}{{\selectfont\char218}}1
{Ы}{{\selectfont\char219}}1
{Ь}{{\selectfont\char220}}1
{Э}{{\selectfont\char221}}1
{Ю}{{\selectfont\char222}}1
{Я}{{\selectfont\char223}}1
}

\usepackage{pgfplots}
\usetikzlibrary{datavisualization}
\usetikzlibrary{datavisualization.formats.functions}

\usepackage{graphicx}
\newcommand{\imgScale}[3] {
\begin{figure}[h!]
\center{\includegraphics[scale=#1]{img/#2}} % height
\caption{#3}
\label{img:#2}
\end{figure}
}

\newcommand{\imgHeight}[3] {
\begin{figure}[h!]
\center{\includegraphics[height=#1]{img/#2}} % height
\caption{#3}
\label{img:#2}
\end{figure}
}

\usepackage[justification=centering]{caption} % Настройка подписей float объектов

\usepackage[unicode,pdftex]{hyperref} % Ссылки в pdf
\hypersetup{hidelinks}
