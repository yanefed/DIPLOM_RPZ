\addcontentsline{toc}{chapter}{ЗАКЛЮЧЕНИЕ}
\chapter*{\makebox[1.0\linewidth]{ЗАКЛЮЧЕНИЕ}}

Курсовой проект включал в себя анализ, проектирование и разработку
базы данных для прогнозирования вероятности задержки рейса в городе--пересадки из--за технического обслуживания.

На базе результатов анализа былы разработаны база данных, хранимая процедура и триггер для работы с базой данных.

Сравнительный анализ в области работы с строками в хранимой процедуре был проведен на основе замеров времени выполнения.
Было установлено, что время выполнения хранимой процедуры зависит от количества обрабатываемых строк --- с увеличением количества строк время выполнения хранимой процедуры линейно увеличивается.
Результаты анализа позволяют сделать выводы о том, что для оптимизации работы с базой данных необходимо уменьшать количество строк, обрабатываемых хранимой процедурой
и использовать другие методы для работы с данными.

Цель курсовой работы была достигнута.
В ходе выполнения работы были выполнены следующие задачи:
\begin{enumerate}[label=\arabic*)]
	\item проведен анализ предметной области;
	\item спроектирована и разработана база данных в соответствии с поставленной задачей;
	\item выбраны средства реализации базы данных и приложения;
	\item проведен сравнительный анализ работы со строками в хранимой процедуре.
\end{enumerate}
