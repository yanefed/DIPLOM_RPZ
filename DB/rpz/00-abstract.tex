\addcontentsline{toc}{chapter}{РЕФЕРАТ}
\chapter*{\makebox[1.0\linewidth]{РЕФЕРАТ}}

Расчетно-пояснительная записка~\pageref{LastPage} с., \totalfigures\ рис., 3 лист., 14 ист., 1 прил.

БАЗА ДАННЫХ, PostgreSQL, CИСТЕМА УПРАВЛЕНИЯ БАЗАМИ ДАННЫХ, Python, SQLAlchemy, РЕЛЯЦИОННАЯ МОДЕЛЬ.

Цель работы: разработка базы данных для прогнозирования вероятности задержки рейса в городе--пересадки из--за технического обслуживания.

Данный курсовой проект включет в себя анализ, проектирование и разработку
базы данных для прогнозирования вероятности задержки рейса в городе--пересадки из--за технического обслуживания.
В качестве системы управления базами данных была выбрана PostgreSQL, а для взаимодействия с базами данных в объектно--ориентированной парадигме была выбрана технология SQLAlchemy с реализацией на языке программирования Python.

На базе результатов анализа были разработаны база данных, хранимая процедура и триггер для работы с базой данных.