\chapter{Сравнение существующих решений}

В качестве критериев сравнения решений задачи регрессии можно выделить следующие характеристики:
\begin{enumerate}[label=\arabic*), leftmargin=1.6\parindent]
    \item возможность реализации без специальных требований;
    \item способность учитывать нелинейные зависимости;
    \item интерпретируемость параметров;
    \item возможность анализа как числовых, так и категориальных данных;
    \item возможность использования для больших объемов данных;
    \item устойчивость к переобучению;
    \item корректная обработка выбросов в данных.
\end{enumerate}

В таблице~\ref{tab:tabl5} приведено сравнение существующих решений задачи регрессии по выделенным критериям.

\begin{table}[H]
    \centering
    \caption{Сравнение существующих решений задачи регрессии}
    \begin{tabularx}{\textwidth}{|>{\centering\arraybackslash}X|>{\centering\arraybackslash}X|>{\centering\arraybackslash}X|>{\centering\arraybackslash}X|}
        \hline
        Критерий & Линейная\newline регрессия & Логистическая\newline регрессия & Адаптивная\newline регрессия \\
        \hline
        1 & Да & Да & Нет \\
        \hline
        2 & Нет & Нет & Да \\
        \hline
        3 & Да & Да & Нет \\
        \hline
        4 & Нет & Нет & Да \\
        \hline
        5 & Да & Да & Нет \\
        \hline
        6 & Да (при регуляризации) & Нет & Нет \\
        \hline
        7 & Нет & Частично & Да \\
        \hline
    \end{tabularx}
    \label{tab:tabl5}
\end{table}

Линейная регрессия отличается высокой интерпретируемостью, однако она ограничена в способности моделировать нелинейные зависимости.
Логистическая регрессия применима в задачах бинарной классификации, но имеет ограничения в условиях высокой размерности данных или сложных нелинейных связей.
Адаптивная регрессия на примере метода МАРС, продемонстрировала высокую гибкость и способность моделировать сложные и нелинейные зависимости, что делает её подходящей для задач с большим количеством переменных и сложными взаимодействиями, однако она требует тщательной настройки для предотвращения переобучения и обеспечения интерпретируемости.