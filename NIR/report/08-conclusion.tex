\chapter*{ЗАКЛЮЧЕНИЕ}
\addcontentsline{toc}{chapter}{ЗАКЛЮЧЕНИЕ}

В данной работе были рассмотрены различные методы регрессии, включая линейную регрессию, логистическую регрессию и адаптивную регрессию.

Цель работы достигнута.

В ходе выполнения работы были решены следующие задачи:
\begin{itemize}
    \item был проведен анализ предметной области;
    \item был проведен обзор существующих решений задачи регрессии;
    \item были сформулированы критерии сравнения решений задачи регрессии;
    \item были классифицированы существующие решения задачи регрессии;
\end{itemize}

В ходе работы было выяснено, что каждый метод регрессии имеет свои преимущества и ограничения, которые делают его более подходящим для конкретных типов задач.

Таким образом, выбор подходящего метода зависит от особенностей данных и задачи.
Линейная регрессия и логистическая регрессия применимы в случае простых линейных зависимостей и бинарных задач, в то время как адаптивные методы регрессии обеспечивают более высокую точность в сложных случаях, требующих гибкости и учёта нелинейных взаимодействий между переменными.