\chapter*{ВВЕДЕНИЕ}
\addcontentsline{toc}{chapter}{ВВЕДЕНИЕ}

Регрессионный анализ по праву может быть назван основным методом современной математической статистики.
Он стал неотъемлемой частью современных методов анализа данных, находя свое отражение в различных подходах, включая методы усреднения, процедуры сглаживания, алгоритмы согласования противоречивых данных и концепции, основанные на принципах оптимальности.
Регрессия --- это квинтэссенция понятия целесообразности~\cite{norman}.

Решение задачи регрессии является ключевым этапом в анализе данных и активно применяется в самых разнообразных областях: от анализа экономических процессов и прогнозирования рыночных тенденций до моделирования сложных физических и инженерных систем.
Такие методы позволяют учитывать нелинейные зависимости, высокую размерность признаков и наличие выбросов, что делает их универсальным инструментом для обработки и интерпретации сложных данных~\cite{bishop}.

Целью работы является проведение анализа методов и подходов к решению задачи регрессии.

Для достижения поставленной цели, необходимо решить следующие задачи:
\begin{itemize}
    \item провести анализ предметной области;
    \item провести обзор существующих решений задачи регрессии;
    \item сформулировать критерии сравнения решений задачи регрессии;
    \item классифицировать существующие решения задачи регрессии.
\end{itemize}
