\chapter{Конструкторский раздел}

%Разработать метод прогнозирования задержек рейсов на основе итеративной адаптации. Изложить особенности предлагаемого метода. Описать основные этапы разрабатываемого метода в виде детализированной IDEF0-диаграммы и схем алгоритмов. Спроектировать программное обеспечение для реализации разрабатываемого метода.

\section{Требования и ограничения к разрабатываемому методу}

К методу прогнозирования задержек рейсов на основе итеративной адаптации с учётом пространственно--временных факторов предъявляются следующие требования:

\begin{enumerate}[label=\arabic*)]
    \item метод должен прогнозировать длительность задержки рейса в минутах с указанием доверительного интервала;
    \item метод должен учитывать пространственно--временные факторы: географическое расположение аэропортов и временные характеристики рейсов;
    \item метод должен обеспечивать итеративную адаптацию модели.
\end{enumerate}

Также представлен ряд ограничений:
\begin{enumerate}[label=\arabic*)]
    \item для корректной работы, метод опирается на исторические данные (последние 90 дней);
    \item качество прогнозирования метода зависит от полноты исторических данных;
    \item метод не учитывает форс--мажорные события.
\end{enumerate}


\section{Требования к разрабатываемому программному обеспечению}

Программное обеспечение должно соответствовать следующим требованиям:

\begin{enumerate}[label=\arabic*)]
    \item реализовать итеративно--адаптивное прогнозирование задержек авиарейсов с использованием комбинации пространственных и временных характеристик;
    \item предоставлять пользовательский интерфейс для получения прогнозов по маршруту и дате с указанием уровня достоверности.
\end{enumerate}


\section{Основные этапы разрабатываемого метода}
\subsection{IDEF0-диаграмма уровня А1}
На рисунке~\ref{img:idef0-a1} представлена диаграмма IDEF0 уровня А1, описывающая процесс прогнозирования задержек.
Входные данные включают исторические записи, метеопараметры и операционные данные аэропортов.
Управляющие воздействия — настройки модели.
Выход — прогноз задержек с оценкой точности.

\includeimage
{idef0-a1}
{f}
{H}
{\textwidth}
{IDEF0--диаграмма уровня А1 процесса прогнозирования задержек}


\subsection{Схемы алгоритмов}

\subsection{Структура разрабатываемого программного обеспечения}

\section*{Выводы}
В рамках данного раздела разработаны требования к методу прогнозирования задержек авиарейсов и соответствующему программному обеспечению.
Основные этапы метода включают сбор данных, обучение адаптивной модели и интеграцию с внешними системами.
Использование IDEF0--диаграмм и алгоритмических схем позволило визуализировать логику работы системы.
Спроектированная архитектура программного обеспечения обеспечивает масштабируемость и возможность внедрения в реальную инфраструктуру аэропортов.
