\chapter*{ВВЕДЕНИЕ}
\addcontentsline{toc}{chapter}{ВВЕДЕНИЕ}

Воздушные перевозки являются одним из наиболее предпочтительных и быстрых видов транспорта.
Самолеты --- это самые известные средства воздушного транспорта и могут использоваться не только для коммерческих~/~некоммерческих пассажирских перевозок, но и для транспортировки грузов и оборудования.
Объем коммерческих авиаперевозок непрерывно растет во всем мире на протяжении десятилетий, поэтому для поддержания или улучшения качества сервиса, необходимо удовлетворять ожидания пассажиров.
Поскольку пунктуальность оказывает влияние на предпочтения пассажиров при выборе авиалиний, крайне важно, чтобы рейсы четко придерживались своего расписания~\cite{tat}.

В современном мире многие компании нередко прибегают к технологиям анализа данных и машинного обучения для оптимизации работы их сервисов и построению прогнозов, исключением не являются и авиакомпании.
Получение своевременной информации о возможных задержках рейсов дает возможность компании перестроить работу в аэропорту, избежать внештатных ситуаций, а также заблаговременно предоставить необходимую информацию пассажирам~\cite{nozdrin}.

Заблаговременное информирование о вероятности задержки рейса пассажиров позволяет последним принимать более рациональные решения, включая выбор альтернативных маршрутов или распределение времени для учета возможных задержек.
Авиакомпании, в свою очередь, могут использовать такие прогнозы для управления ожиданиями пассажиров и реализации мер по снижению потенциальных негативных воздействий на их путешествия~\cite{trt}.

Целью работы является анализ существующих методов прогнозирования, а также разработка системы, способной проводить данное прогнозирование.

Для достижения цели необходимо выполнить следующие задачи:
\begin{enumerate}[label=\arabic*)]
    \item провести анализ предметной области;
    \item разработать метод прогнозирования задержек рейсов в соответствии с поставленной задачей;
    \item выбрать средства программной реализации;
    \item провести сравнительный анализ точности прогнозирования по ключевым метрикам.
\end{enumerate}

%Часть 1. Аналитический раздел
%Провести анализ предметной области прогнозирования задержек авиарейсов с учётом пространственно-временных факторов. Провести обзор существующих методов прогнозирования, привести результаты сравнительного анализа. Сформулировать цель работы и формализовать постановку задачи в виде IDEF0-диаграммы.
%
%Часть 2. Конструкторский раздел
%Разработать метод прогнозирования задержек рейсов на основе итеративной адаптации. Изложить особенности предлагаемого метода. Описать основные этапы разрабатываемого метода в виде детализированной IDEF0-диаграммы и схем алгоритмов. Спроектировать программное обеспечение для реализации разрабатываемого метода.
%
%Часть 3. Технологический раздел
%Обосновать выбор средств программной реализации. Описать формат входных и выходных данных. Разработать программное обеспечение, реализующее предложенный метод. Описать взаимодействие пользователя с программным обеспечением.
%
%Часть 4. Исследовательский раздел
%Привести примеры работы разработанного метода на исторических данных авиарейсов. Оценить точность прогнозирования по ключевым метрикам. Проанализировать влияние различных факторов на результаты прогнозирования и сформулировать выводы.