\chapter{Технологический раздел}

%Обосновать выбор средств программной реализации. Описать формат входных и выходных данных. Разработать программное обеспечение, реализующее предложенный метод. Описать взаимодействие пользователя с программным обеспечением.

\section{Выбор средств программной реализации}

В качестве языка программирования для реализации предлагаемого метода был выбран Python.

Выбор обусловлен тем, что данный язык имеет широкий набор библиотек для работы с данными, включая библиотеки для обработки временных рядов, машинного обучения и визуализации данных.

\section{Выбор среды разработки}

В качестве среды разработки был выбран PyCharm, так как он предоставляет множество инструментов и функций для разработки, отладки и выполнения кода на Python.
Данная среда разработки предоставляет такие компоненты, как текстовый редактор, панель переменных окружения, история команд, интеграция файловой системы, а также встроенные инструменты для работы с базами данных и системами контроля версий.

\section{Выбор базы данных}

В качестве базы данных для хранения исторических данных о рейсах был выбран PostgreSQL.

Выбор обусловлен тем, что PostgreSQL является реляционной базой данных с открытым кодом, обеспечивающую поддержку расширенных типов данных и возможность работы с большими объемами информации.

\section{Формат входных и выходных данных}

Входные данные для разрабатываемого метода включают:
\begin{itemize}[label=---]
    \item аэропорт вылета;
    \item аэропорт назначения;
    \item авиакомпания;
    \item дата и время вылета.
\end{itemize}

Выходные данные для разрабатываемого метода включают:
\begin{itemize}[label=---]
    \item прогноз задержки рейса в минутах;
    \item уверенность прогноза в процентах;
    \item график задержек рейсов за последние 7 дней.
\end{itemize}

\section{Реализация предлагаемого метода}
\section{Описание взаимодействия пользователя с программным обеспечением}
\section{Сборка и запуск проекта}
